\section{ETCCDI Climate Change Indices}

\begin{frame}
\frametitle{Indici ETCCDI}
Il \textbf{Expert Team on Climate Change Detection and Indices (ETCCDI)} ha definito 27 indici climatici focalizzati su eventi estremi. \\
\vspace{0.5cm}
Gli indici sono definiti in base alla precipitazione e temperatura giornaliera, e si concentrano sui cambiamenti di intensità, durata e frequenza degli eventi climatici estremi.
\end{frame}


\begin{frame}
  \frametitle{Indici ETCCDI (27)}
  \setbeamercolor{block title}{use=structure,bg=MidnightBlue,      fg=white}
  \setbeamercolor{block body} {use=structure,bg=MidnightBlue!10!white, fg=black,}

	\begin{block}<1->{Absolute (9)}
	  Temperatura del giorno più caldo o più freddo dell'anno, o il massimo annuale di precipitazione.
	\end{block}


	\begin{block}<2->{Threshold (7)}
    Numero di giorni in cui viene superata una soglia fissa di temperatura o di precipitazione.
  \end{block}


	\begin{block}<3->{Duration (5)}
	 Durata dei periodi piovosi, secchi, caldi e freddi.
  \end{block}
  
    \begin{block}<4->{Percentiles (6)}
     Percentuali di superamento al di sopra o al di sotto del 10° o 90° percentile. 
     %derivato dal periodo di riferimento.
  \end{block}
  
  
\end{frame}